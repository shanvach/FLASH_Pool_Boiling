\documentclass{article}
\usepackage{graphicx}   % need for figures
\usepackage{subfigure}  % use for side-by-side figures
\setlength{\oddsidemargin}{0mm}
\setlength{\evensidemargin}{0mm}

\pagestyle{empty} 
\begin{document}

\begin{figure}[h!]
\subfigure[]{
\includegraphics[angle=270,width=74mm]{He.ps}}
\subfigure[]{
\includegraphics[angle=270,width=74mm]{C.ps}}
\subfigure[]{
\includegraphics[angle=270,width=74mm]{N.ps}}
\subfigure[]{
\includegraphics[angle=270,width=74mm]{O.ps}}
\subfigure[]{
\includegraphics[angle=270,width=74mm]{Ne.ps}}
\subfigure[]{
\includegraphics[angle=270,width=74mm]{Mg.ps}}
\caption{\label{fig:1} FLASH2 neitest results.}
\end{figure}



\begin{figure}[h!]
\subfigure[]{
\includegraphics[angle=270,width=74mm]{Si.ps}}
\subfigure[]{
\includegraphics[angle=270,width=74mm]{S.ps}}
\subfigure[]{
\includegraphics[angle=270,width=74mm]{Ar.ps}}
\subfigure[]{
\includegraphics[angle=270,width=74mm]{Ca.ps}}
\subfigure[]{
\includegraphics[angle=270,width=74mm]{Fe.ps}}
\subfigure[]{
\includegraphics[angle=270,width=74mm]{Ni.ps}}
\caption{\label{fig:2} FLASH2 neitest continued.}
\end{figure}

\end{document}
